\section{Conclusion}\label{sec:conclusion}

To conclude, we will answer the initial research questions based on the analyses presented above and discuss some limitations and problems of our project.

\begin{questions}
    \item \textbf{Do song lyrics of German rap in general possess a negative sentiment?} \\
    Due to the limited evaluation results of the pre-trained models (German Sentiment Bert, Toxicity, Zero-shot Classification), the question whether a negative sentiment generally prevails in German rap can only be adequately answered based on the occurrences check. However, for the occurrences check, it can be stated that a general negative sentiment does not necessarily apply to the songs of the German raps. The comparison of the three most prevalent categories love, violence and misogyny in the occurrences check shows that all three categories have comparatively high occurrences in the lyrics, so the category love represents an opposition to the others. 
    \item \textbf{Does hate, discrimination \& racism exist in German rap song lyrics?} \\
    All four different methods of our analysis indicate the actual presence of hateful, discriminatory and racist topics in the lyrics. In particular, the occurrences check shows high occurrences of the categories violence and misogyny, the pre-trained models underline this assumption either. However, it should be noted that the evaluation of these models was weak.
    \item \textbf{How prevalent is hate, discrimination \& racism in German rap song lyrics?} \\
    A precise quantification of the presence of negatively associated topics in German rap songs is difficult on the basis of the research carried out. However, the occurrences check shows that about 50\% of German rap songs contain themes of violence, about 41\% themes of misogyny (see \autoref{sec:results}). Over time, it seems that violence, discrimination and racism play a decreasing role in German rap.
\end{questions}

During our work on the project, we encountered several difficulties that limited our ability to work on the songs correctly.

One of the main problems we encountered was the way the lyrics are written. The language used in German rap is special: slang, juvenile words, foreign words and the phonetic way of writing and spelling lyrics characterise this genre. All this made the pre-processing of these texts difficult using natural language processing tools such as lemmatisation and tokenisation. 

As an example, many words are written as they are spoken, i.e. 'ich hab' statt 'ich habe'. Another example is the word 'eine', which is often written as 'ne'. In other cases, nouns were not written with a capital letter, which made it difficult to distinguish between a verb and a noun.  Because of these deviations from the correct spelling of words, lemmatisation or stop word removal does not work, i.e. the word 'ne' wouldn't be removed because it is not recognised as a stop word. We tried to overcome this issue by using regex, but it was impossible to account for all cases. In our experiments, we found that words that were not supposed to be affected by the various regex rules were changed etiher, making this option useless. This limitation eventually made it impossible to learn the vocabulary of these songs with methods like Word2Vec accurately. The same words were treated as two different words depending on how they were spelled, i.e. 'hab' and 'habe' were considered as two different words.

A further difficulty that may have limited our results even more was the lack of punctuation. For correct processing of the texts, whole songs should be correctly divided into different sentences. The lack of punctuation in most of the songs prevented us from understanding which word belonged to which part of the lyrics. We were able to solve this problem with the automatic punctuation model mentioned in the pipeline section. Without this solution, it would have been very difficult to process the songs, even with pre-trained models.

The biggest lesson we learnt from this project, due to the mentioned limitations and the resulting weaknesses in the results, is that the quality of the data is of vital importance for the success of a text analytics project. 

\newpage