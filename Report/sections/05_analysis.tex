\section{Analysis}\label{sec:analysis}

In the following, the results listed in \autoref{sec:results} will be put into the context of the research questions and will be interpreted. 

The analysis of the Occurrences Check shows that particularly violence and misogyny are an existing problem in the genre of German rap. As \autoref{fig:occurrences_time_series} shows, occurrences of this category are present throughout the entire period of the study. This proposition is further supported by the fact that 51\% of all the songs studied contain violent terms in some form. However, at the same time it must be taken into account that the category love has a similar behaviour and therefore represents an antithesis to the problem described. 61\% of the songs studied have occurrences on the theme of love. Overall, these observations can be interpreted in the sense that emotionality plays a large role in German rap.

Furthermore, the analysis of the time series of occurrences check shows a slight decrease in the most prevalent categories misogyny, violence and mysogyny from 2005 onwards (see \autoref{fig:occurrences_time_series}). We suppose that these changes in German rap are related to transformations within German society, i.e. changes in terms of gender equality due to movements like MeToo-discussion, open-mindedness and acceptance towards gay community or changes in attitudes towards minorities. These transformations are not only due to evolving politics, but can also be seen internationally and especially in the US, which has a huge international influence on rap in general. As society adopts these attitudes and societal values, German rap artists might also become aware of the importance of their lyrics and the message they convey, which requires artists to adapt to these social changes. Another possible reason would be the constantly evolving vocabulary and slang, which could mean that our dictionaries and the words they contain fit better in the early 2000s than in later times.

The application of the sentiment analysis via the German Sentiment Bert and Toxicity-Model allows only limited interpretative space than the occurrences check due to relatively weak evaluation results. Generally, the two models show a strong one-sided allocation, which is shown both in the observation of the distributions via histogram (see \autoref{fig:sentiment}, \autoref{fig:toxicity}) and in the confusion matrices (see \autoref{tab:confusion_matrix_sentiment}, \autoref{fig:toxicity}). Of the 70 songs evaluated, 26 were manually classified as positive, but the classifier recognised only 2, respectively 8\%, as such. Similarly, the recall for the toxicity classification is 6\% for the classification 'toxic', since only very few songs were actually classified as such by the classifier. Because of this bias, the values for the recall of the respective contrary class are very high. The precision is also of limited significance due to the fact that only two songs were classified as toxic and three as positive. In particular, the value 1.0 for the category neutral in the toxicity classifier is therefore meaningless. Overall, the rather sobering results of the evaluation can be summed up by the low macro average: This is 0.45 for the German Sentiment Bert and 0.45 for the Toxicity classifier. An improved evaluation that takes into account more data from the respective under-represented class could provide improved insights.

Regardless of the results of the evaluation, at least the bias of the data of the sentiment analysis of the songs towards rather negative sentiment could be explained to a certain extent. As mentioned before, emotions very often play a major role in songs, which are formulated in a particularly expressive way compared to natural language - whether on the topic of love, hate, etc. This could cause the sentiment model to increasingly detect emotions and then evaluate them as negative, as this is not frequently found in ordinary language usage.

Comparing the results of German Sentiment Bert and the Toxicity model, we obtained a negative Pearson correlation of -0.13 (see \autoref{sec:results}), which means that there is little to no noticeable correlation between the two models. This means that at least some songs were classified as negative but not toxic or positive but toxic. These results are in line with our expectations and raise interesting questions: 'What leads to the decision to rate a song as positive or negative?' 'Does the use of swear words mean that the message conveyed is necessarily toxic?' 

It is possible to convey a positive message but use offensive words, while it is also possible to make a negative statement about something without being toxic. This is one of the difficulties with sentiment analysis and the way humans or a machine can interpret a text: Many factors can influence the decision, e.g. social prejudices, personal experience, gender, etc. In terms of natural language processing models, this means that the dataset on which the model has been trained has a big impact on the interpretation obtained by the model. 

The results of the zero-shot classification should also be treated with caution due to imprecise evaluation results (see \autoref{tab:confusion_matrix_zero_shot}). However, it must be taken into account that the classification of eight different labels is significantly more difficult than for binary models. Accordingly, the weighted average of the F1-score of 0.52 is not ideal, but also not completely poor. The macro average of the F1-score, however, is rather weak at 0.37. Apart from the classes 'homophobic' and 'racist', which the model did not see as the most probable in any of the 5992 songs examined, the zero-shot classifier at least partially recognises the identical labels as we humans did. The absence of the classes 'homophobic' and 'racist' can be explained by a correlation of the classes to the category 'violent', which was always considered more relevant. Over all categories, the classification into 'affectionate' and 'sad' was the most accurate among all classes.

Based on \autoref{fig:zero-shot} and \autoref{fig:zero-shot2}, it can only be stated that violence at least does not seem to be completely irrelevant in the songs studied. The high proportion of songs classified as positive and friendly might not correspond entirely to reality, taking into account the various occurrences of terms with negative connotations in the context of the occurrences check.

