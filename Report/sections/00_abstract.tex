\section*{Abstract}\label{sec:abstract}
Rap music has become an essential part of pop culture worldwide and culture in general, also in Germany. German rap plays a major role in the German music scene, with the top three song artists of 2021 in Germany being rap artists. In addition to its popularity, German rap is known for its use of profanity and vulgarity, for its harsh lyrics and for the very sensitive topics it addresses. In this work, we wanted to investigate to what extent the accusations regarding the high level of violence, hatred and discrimination in German rap can be confirmed using methods of text analysis by machine. For this purpose, we collected data from 5992 songs of German rap artists in the years 1998 to 2022. We counted occurrences of specific terms of different categories of hate and discrimination, applied models of sentiment and toxicity analysis and also tried to assign songs to different classes via zero-shot classification. We were able to identify a tendency towards negative expressions in German rap, but at the same time it proved difficult to apply the methods of natural language processing. This was due to the fact that German rap songs usually do not follow the common syntactic scheme of German language.