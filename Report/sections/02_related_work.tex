\section{Related Work}\label{sec:research}

In 2018, the two known German Rappers 'Kollegah' and 'Farid Bang' have won a ECHO-Prize, despite their antisemitic text lines (see Feuerbach, \cite{kollegahfarid}).
This implies the significant and popularity of German rap in German society, despite its negative image and very aggressive nature. It is also worth to note, that Gangsta-Rap is very popular especially among young people and that it has been studied and found to have a negative influence on them \cite{jugendeinfluss, bielefeld_studie,salomo_greven_2021}

Various journalistic and social science works in the past have dealt with the role of German rap in society.

The beginning of the 2000s marks a significant increase in the amount of German rap texts containing vulgarity, misogyny, sexism, anti-Semitism and violence. From about 4-5\% of German rap songs containing sexist terms, the 2000s marked a jump towards ca. 25\% of the songs containing such terms. Between 2005 and 2013 the trend has declined only to later on in 2018 go up again. An explanation for this might be, that sexism in rap songs has become more subtle by using less sexist terms but at the same time they still promote the sexist image and is also harder to detect by listeners as much as by means of text analysis. \cite{rohwer_2020}

This general rise of this very violent/hatred-focused rap is directly connected to the rise of the 'gangster rap', which has become the most successful sub-genre of rap in general (like in the USA) and in German rap in particular.
Gangster-Rap concentrates mostly on on the so called prison-culture. In the lyrics of such songs, one encounters very often terms related to violence, drugs, segregation from other (social) groups. It also conveys the hardships of being a minority in Germany and puts a spotlight on the socially weaker. \cite{wiegangsta}



In addition to sociotechnical analyses, there are two data-driven approaches to analyze the song lyrics of various German rappers. In 2016, Bayerischer Rundfunk's cultural magazine Puls \cite{puls_2016} examined the political correctness of various song lyrics by German rappers, using a very similar methodology to the one we will use in this paper. Puls selected the five most commercially successful albums by German rappers in each year for the period 2006 to 2016 and downloaded the song lyrics via Genius. These song lyrics were examined for specific discriminatory word groups - with a particular focus on homophobic, racist, misogynistic, and ableist terms.

Puls observed that the use of discriminatory language increased over the first part of the sample period and decreased towards the end. Misogynistic and homophobic remarks played a particularly significant role. Discrimination against the disabled was also a permanent feature of the song lyrics studied, while racism was rather less prevalent. The author of the study also emphasizes the lower significance of the study due to the limitation to five albums per year.



In contrast to the analyses of Puls, we would like to get a broader view of the sentiment of German rap. Concretely, this means that we want to include data from more artists and songs in our analysis. In addition, we do not only want to consider frequencies of certain words, but more in-depth methods of text analysis, which are based on machine learning. Generally, the goal of this project is to gain as much information as possible about song lyrics and to determine their 'fairness' in social context. The insights of this project could also be extended to other music genres. In addition, the focus of this project is on German language, song lyrics in English could also be analyzed with the same approach.

