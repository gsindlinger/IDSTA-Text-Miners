\section{Related Work}\label{sec:research}

Various journalistic and social science works in the past have dealt with the role of German rap in society as work from Lena Gauer \cite{gauer}, Melanie Heinisch \cite{heinisch}, Markus Klein \cite{klein2011dies} or Tobias Wiese \cite{wiese2021identitat}. 

Michael Huber \cite{wiegangsta} explains the popularity of the rap genre in the context of the rise 'Gangsta Rap', which finds its origin in the United States. According to Huber, 'Gangsta Rap' in Germany usually focuses on the so-called prison culture. In the lyrics of such songs, one very often encounters terms that deal with violence, drugs, separation from other social groups. It also conveys the hardships of being a minority in Germany and focuses on the socially weaker.

In addition to sociotechnical analyses, there are two data-driven approaches to analyze the song lyrics of various German rappers. In 2016, Bayerischer Rundfunk's cultural magazine Puls \cite{puls_2016} examined the political correctness of various song lyrics by German rappers, using a very similar methodology to the one we will use in this paper. Puls selected the five most commercially successful albums by German rappers in each year for the period 2006 to 2016 and downloaded the song lyrics via Genius. These song lyrics were examined for specific discriminatory word groups - with a particular focus on homophobic, racist, misogynistic, and ableist terms.

Puls observed that the use of discriminatory language increased over the first part of the sample period and decreased towards the end. Misogynistic and homophobic remarks played a particularly significant role. Discrimination against the disabled was also a permanent feature of the song lyrics studied, while racism was rather less prevalent. The author of the study also emphasizes the lower significance of the study due to the limitation to five albums per year.

Another quantitative analysis of the situation in Deutschrap is provided by Spiegel magazine \cite{rohwer_2020} in 2020. Author Bj{\"o}rn Rohwer concludes the following: The beginning of the 2000s marks a significant increase in the amount of German rap texts containing vulgarity, misogyny, sexism, anti-Semitism and violence. From about 4-5\% of German rap songs containing sexist terms, the 2000s marked a jump towards ca. 25\% of the songs containing such terms. Between 2005 and 2013 the trend has declined only to later on in 2018 go up again. An explanation for this might be, that sexism in rap songs has become more subtle by using less sexist terms but at the same time they still promote the sexist image and is also harder to detect by listeners as much as by means of text analysis.

In contrast to the analyses of Puls and Spiegel, we wanted to get a broader view of the sentiment of German rap. Concretely, we did not only consider frequencies of certain words, but more in-depth methods of text analysis, which are based on machine learning. In addition to that, we also included data from more artists and songs in our analysis compared to Puls. Generally, the goal of this project was to gain as much information as possible about song lyrics and to determine their 'fairness' in social context.












