\section{Research Topic Summary}\label{sec:research}

Various journalistic and social science works in the past have dealt with the role of German rap in society.

\textbf{TODO: Summarize and search for research projects in this field! These are only some suggestions I could find.
\cite{heinisch2018schlampe},
\cite{ahlers_2019},
\cite{wiegel2011deutscher}
journalistic:
\cite{rohwer_2020}}

In addition to socio-technical analyses, there is one data-driven approach to analyze the song lyrics of various German rappers. In 2016, Bayerischer Rundfunk's cultural magazine Puls \cite{puls_2016} examined the political correctness of various song lyrics by German rappers using a very similar methodology to the one we will use in this paper. Puls selected the five most commercially successful albums by German rappers in each year for the period 2006 to 2016 and downloaded the song lyrics via Genius. These song lyrics were examined for specific discriminatory word groups - with a particular focus on homophobic, racist, misogynistic, and ableist terms.

Puls observed that the use of discriminatory language increased over the first part of the sample period and decreased towards the end. Misogynistic and homophobic remarks played a particularly significant role. Discrimination against the disabled was also a permanent feature of the song lyrics studied, while racism was rather less prevalent. The author of the study also emphasizes the lower significance of the study due to the limitation to five albums per year.

In contrast to the analyses of Puls, we would like to get a broader view of the sentiment of German rap. Concretely, this means that we want to include data from more artists and songs in our analysis. In addition, we do not only want to consider frequencies of certain words, but more in-depth methods of text analysis, which are based on machine learning. Generally, the goal of this project is to gain as much information as possible about song lyrics and to determine their 'fairness' in social context. The insights of this project could also be extended to other music genres. In addition, the focus of this project is on German language, song lyrics in English could also be analyzed with the same approach.

