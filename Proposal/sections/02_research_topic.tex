\section{Research Topic Summary}

Various journalistic and social science works in the past have dealt with the role of German rap in society.

\cite{heinisch2018schlampe},
\cite{ahlers_2019},
\cite{wiegel2011deutscher}

In addition to socio-technical analyses, there is one data-driven approache to analyze the song lyrics of various German rappers. In 2016, Bayerischer Rundfunk's cultural magazine Puls \cite{puls_2016} examined the political correctness of various song lyrics by German rappers using a very similar methodology to the one we will use in this paper. Puls selected the five most commercially successful albums by German rappers in each year for the period 2006 to 2016 and downloaded the song lyrics via Genius. These song lyrics were examined for specific discriminatory word groups -- with a particular focus on homophobic, racist, misogynistic, and ableist terms.

Puls observed that the use of discriminatory language increased over the first part of the sample period and decreased towards the end. Misogynistic and homophobic remarks played a particularly significant role. Discrimination against the disabled was also a permanent feature of the song lyrics studied, while racism was rather less prevalent. The author of the study also emphasizes the lower significance of the study due to the limitation to five albums per year.

What we do differently:
- More albums / song lyrics
- Use of pretrained/self-trained (optional) machine learning frameworks for sentiment analysis, hate speech detection --> we want to pay attention to whether particularly fair lyrics are also present
- Advanced analysis of terms using Elastic Search

