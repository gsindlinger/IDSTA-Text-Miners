\section{Motivation}
'I leave no whore daughter unfucked, everyone wants my dick -- even lesbians get turned around!' -- Excerpts from song lines by German rappers such as Bausa \cite{steffes-lay_2019} provide material for discussion in German society and pose the question of how far artistic freedom can go in music and where insurmountable boundaries are crossed. Whether homophobia \cite{steffes-lay_2019}, misogyny \cite{steffes-lay_2019} or antisemitism \cite{salomo_greven_2021}, in the public perception German rap seems to be one thing above all: Harsh and unfair. The popularity and sales figures of German rappers, on the other hand, justify their song texts and actings: at the end of October 2022, there were a total of ten titles in the top 20 singles charts in Germany that can be assigned to the genre of German rap \cite{mtv_germany_2022}. And in 2021, rapper Capital Bra was the most successful German musician in terms of the number of different number 1 hits \cite{br_2019}. 

Contrary to the general negative impression, there are many attempts by artists who oppose against the negative image of rap in Germany with their lyrics and actions \cite{Deutschlandfunk_2021}. Some artists use their songs also used to specifically address socio-critical issues - such as the 'Black Lives Matter' movement, police violence or the integration of refugees \cite{me-redaktion_2021}.

In this paper we would like to investigate the controversial debate around German Rap in an analytical manner. For this purpose, the song lyrics of various successful rappers of the genre of German rap will be analyzed on the basis of methods of textual data science. The following questions will be focused:

\begin{questions}
    \item \textbf{Do song lyrics of German rap in general possess a negative sentiment?}
    \item \textbf{Does hate, discrimination \& racism exist in German rap song lyrics?}
    \item \textbf{How prevalent is hate, discrimination \& racism in German rap song lyrics?}
\end{questions}










